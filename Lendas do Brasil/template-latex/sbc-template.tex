\documentclass[12pt]{article}

\usepackage{sbc-template}
\usepackage{graphicx,url}
\usepackage{url}
\usepackage[utf8]{inputenc}
\usepackage[brazil]{babel}
\usepackage[latin1]{inputenc}  

     
\sloppy

\title{Lendas do Brasil}

\author{Jairton Felix Cavalcante de Melo\inst{1} }


\address{Instituto Federal de Educação, Ciência e Tecnologia do Ceará (IFCE)\\
  Maracanaú -- CE -- Brasil
}

\begin{document} 

\maketitle

\begin{abstract}
  This following text represents the result of readings carried out on development concepts and techniques involving the creation of a digital game, as well as comparative analyzes of games whose aspects coincide with those that will be proposed for the project, all done in order to start the process of creating a game called Lendas do Brasil, a digital board game using the theme of Brazilian folklore that will be used as a conclusion work for the computer science course at the Instituto de Ciências e Tecnologias do Ceará (IFCE).
\end{abstract}
     
\begin{resumo} 
 O presente texto representa o resultado de leituras realizadas a respeito dos conceitos e técnicas de desenvolvimento envolvendo a criação de um jogo digital, assim como análises comparativas de jogos cujos aspectos coincidem com aqueles serão proposto para o projeto, tudo isso feito a fim de iniciar o processo de criação de um jogo  chamado Lendas do Brasil, um jogo de tabuleiro digital com temática do folclore brasileiro que será utilizado como trabalho de conclusão de curso do curso de ciências da computação do Instituto de Ciências e Tecnologias do Ceará (IFCE).
\end{resumo}


\section{Introdução}

Segundo o conceito de Loureiro et al (2015) um jogo pode ser definido como um sistema
formal de regras onde o jogador está emocionalmente ligado ao resultado do seu esforço em lidar com esse conjunto de regras. Com o passar do tempo, os jogos foram evoluindo e se tornaram digitais e conquistaram um espaço importante na vida de crianças, jovens e adultos, segundo a ideia desenvolvida por de Almeida Reis e Cavichiolli (2008), os jogos digitais são como muitas outras atividades de lazer, nas quais a busca da excitação  é o principal atrativo e a fonte do prazer, contudo sua diferença é que os jogos eletrônicos contam com a comodidade como vantagem. 

Todavia, nos dias atuais, jogos eletrônicos tem assumido outros papéis além de trazerem entretenimento, podendo serem utilizados também para o aprendizado.  Ulbricht e Ravi (2008) dizem que os jogos educacionais, por proporcionarem práticas educacionais atrativas e inovadoras, onde o aluno tem a chance de aprender de forma mais ativa, dinâmica e motivadora, podem se tornar auxiliares importantes do processo de ensino e aprendizagem.

Segundo Roberto (2011), o conceito da palavra folclore fora formado a partir das palavras “folk” que significa povo e “lore” que significa saber, logo a palavra folclore significaria algo como sabedoria do povo, o folclore brasileiro, em especial, é formado pela mistura de atributos das culturas de maioria portuguesas, africanas e indígena, e aprender sobre ele pode gerar curiosidade sobre os antepassados e estimular o interesse pela riqueza cultural de cada região do país.

Tendo em vista isso se propõe construir um jogo de tabuleiro digital, chamado Lendas do Brasil, com perguntas e curiosidades sobre o folclore nacional, de modo a ensinar e despertar o interesse dos jogadores a aprender mais sobre as lendas, cantigas e festas temáticas do país.

\section{Objetivos}

\subsection{Objetivo Geral}

Desenvolver um jogo de tabuleiro digital, através dos conhecimentos adquiridos ao longo do curso e do estudo de trabalhos semelhantes, para apresentar e induzir o interesse ao folclore nacional.

\subsection{Objetivo Específico}

•	Criar um game design canvas de modo definir a estrutura básica que irão compor o jogo proposto.

•	Criar um banco de curiosidades e perguntas integrado numa plataforma.

•	Avaliar a utilização do aplicativo, pontuando sua usabilidade.

•	Auxiliar no conhecimento sobre o folclore nacional.


\section{Referencial Teórico}

\subsection{Folclore}

De acordo com Roberto (2011) a palavra folclore provém do neologismo inglês folk-lore (saber do povo) cunhado por Williem John Thoms, em 1846, para denominar a área de estudos dos pesquisadores que estudavam tradições populares. Todavia, logo após a definição da área começaram as discursões sobre os limites que envolviam o tema, para alguns, por exemplo, a cultura material, tais como esculturas e culinária, estava excluída, mas para outros a cultura material somente estaria integrada ao folclore quando estivesse ligada à cultura não-material,  tais como o estudo das festas tradicionais incluiria a sua culinária.

A Unesco, entidade vinculada à Organização das Nações Unidas (ONU), reconhece o folclore como Patrimônio Cultural Imaterial e ressalta a importância da preservação das diferentes manifestações que o formam, assim, com o término do segunda guerra mundial, a Unesco, liderou um movimento que procurou implantar mecanismos para documentar e preservar tradições que, avaliavam, estariam em vias de desaparecimento e, com isso, nasceu a Comissão Nacional de Folclore (BrasilEscola).

Segundo Delbem (2007), em 1951, ocorreu no Rio de janeiro o 1° Congresso Brasileiro de folclore, onde muitos folcloristas debateram sobre as características que foram atribuídas ao folclore, são elas o autor anônimo, a transmissão oral, ou seja, o aprendizado no folclore ocorreria, exclusivamente, pela fala, a tradicionalidadeo ou antiguidade, a sobrevivência e o conceito de civilidade dos povos, ou seja, como era estruturado os povos na época de criação daquele folclore. Apesar de essas serem as características estabelecidas, elas não são unanimes e já sofreram alterações, até hoje, não só no Brasil, mas em todo o mundo, ainda existem debates sobre o assunto.

Fernades (2003) destaca que o folclore permite observar fenômenos que lançam enorme luz sobre o comportamento humano, como a natureza dos valores culturais de uma coletividade, as circunstancias ou condições em que eles se atualizam, além de também exercer uma função socializadora, principalmente nas crianças.

Um pensamento similar é citado por Sakai (2017), que defende que é importante o ensino do folclore desde a infância, pois contribui na formação do espírito de cidadania e de nacionalidade do aluno, é dito que o folclore reflete o valor moral de uma determinada sociedade, além de produzir um forte sentimento em relação à cultura e à unidade e, também, inspiram pensamento simples e lógico.


\subsection{Jogo}

Segundo o conceito de Loureiro et al (2015) um jogo pode ser definido como um sistema formal de regras onde o jogador está emocionalmente ligado ao resultado do seu esforço em lidar com esse conjunto de regras.

De acordo com Juul (2005), os jogos podem ser agrupados em duas categorias dependendo de como os desafios são apresentados aos jogadores, são eles emergence (Emergente) e progression (Progressivo).  Jogos emergentes são aqueles cujos desafios são apresentados com um número pequeno de regras, de forma simples e dão origem a numerosas variações, exemplos desses tipos de jogos são tabuleiros, cartas e atléticos. Já os progressivos são aqueles que apresentam os objetivos na forma de uma sequência de ações que os jogadores precisam realizar

Crawford(1982) evidencia quatro elementos fundamentais de jogos, são eles: representação, interação, conflito e segurança.

•	Representação: um jogo representa uma visão simplificada e subjetiva da realidade, tendo todo um conjunto de regras já bem definido e nenhum elemento presente nele necessita de referencia ao mundo externo.

•	Interação: em um jogo o expectador é capaz de provocar alterações e verificar suas consequências a cada jogada.

•	Conflito: o jogador busca ativamente atingir algum objetivo, contudo existirão obstáculos, que podem aparecer de diversas formas e que irão atrapalhar o jogador a atingir esse objetivo.

•	Segurança: em um jogo existem regras e consequências para suas ações, contudo tudo se limita a realidade dentro do jogo, sem consequências na vida real.

\subsection{Jogos digitais}

Segundo Lucchese e Ribeiro (2009) um jogo eletrônico é uma atividade lúdica formada por ações e decisões que resultam numa condição final. Tais ações e decisões são limitadas por um conjunto de regras e por um universo, que no contexto dos jogos digitais, são regidos por um programa de computador. No mais, é dito que um jogo é feito de três conceitos sendo eles o enredo, que define não só o tema, mas também os objetivos do jogo e a sequência com a qual os acontecimentos surgem, o motor do jogo, mecanismo que controla a reação do ambiente as ações e decisões do jogador e a interface interativa, que permite a comunicação entre o jogador e o motor do jogo. Por fim, é dito pelos autores que esses jogos podem ser divididos em oito categorias, são elas: estratégia, simuladores, aventura, infantil, esporte, RPG (Role playing games), passatempo e educacional.

\subsection{Jogos Educativos}

De acordo com Luchese e Ribeiro (2009), quando preparados para o contexto educacional os jogos digitais podem receber diferentes nomenclaturas. As mais comuns são jogos educacionais ou educativos, jogos de aprendizagem ou jogos sérios (serious games), sendo que alguns tipos de simuladores também podem ser considerados jogos educacionais.

Jogos podem se utilizar do interesse pelo lúdico de uma população para promover ambientes de aprendizagem atraentes e gratificantes, constituindo-se em uma poderosa ferramenta de estímulo para o desenvolvimento integral de alguém. O motivo de usar o lúdico na aprendizagem ocorre, pois propicia uma sensação de prazer, que é bem recebido pelo público de qualquer faixa etária (Falkembach et al, 2008), assim as pessoas aprendem enquanto se divertem.

Gros (2003) afirma que jogos, para serem utilizados com fins educacionais, precisam ter objetivos de aprendizagem definidos e ensinar conteúdos das disciplinas aos usuários, ou então, promover o desenvolvimento de estratégias ou capacidade cognitiva e intelectual dos jogadores.

Savi e Ulbricht (2008) ressaltam alguns benefícios que os jogos digitais educacionais podem trazer ao processo de aprendizagem, dentre eles temos, funcionar como facilitador de aprendizado, desenvolvimento de habilidades cognitivas, o aprendizado por descoberta, a socialização e o desenvolvimento da coordenação motora.


\subsection{Jogos de Tabuleiro}

Luchese e Ribeiro (2009) citam que um jogo de tabuleiro compreende um plano jogável e delimitado em determinados espaços, e peças que podem ser movidas durante o jogo, sendo que cada peça representa um jogador na partida. Os objetivos de cada jogo de tabuleiro, assim como as estratégias, podem variar dependendo dos objetivos e das regras, todavia, estes jogos geralmente podem ser representados por jogos dinâmicos de informação perfeita.

De acordo La Carretta(2018) jogos de tabuleiro se diferencia de jogos digitais através das seguintes características, estabelecem relações pessoais de forma mais clara entre os jogadores, unem pessoas de diferentes gerações para jogar, podem ser jogados em qualquer lugar, dentre outros.

\section{Estado da Arte}

Tendo em vista a proposta descrita foram selecionados alguns exemplos de jogos que possuem algumas similaridades com o projeto.

Pereira et al (2009) apresenta uma preocupação ao fato apresentado de que poucos alunos se interessam pela disciplina de física que é ensinada no ensino médio, o que se reflete no número de licenciados formados nessa disciplina, os autores citam que os alunos necessitam de novas metodologias e novas técnicas que despertem o interesse pela disciplina, diante disso, eles optaram por criar um jogo, pois segundo eles um jogo é uma atividade rica e de grande efeito que responde às necessidades lúdicas, intelectuais e afetivas, estimulando a vida social e representando, assim, importante contribuição na aprendizagem. Assim, foi criado o jogo educativo “Conhecendo a Física” um jogo de tabuleiro de perguntas e respostas, onde os jogadores devem percorrer as casas do circuito fechado, cumprindo determinados desafios que o jogo exige, nele vence aquele jogador completar o circuito primeiro.

Oliveira et al (2013) cita que o ensino deve ser trabalhado em sala de aula de forma mais dinâmica e divertida, quebrando o hábito escolar da aula tradicional e chamando a atenção do aluno e nesse quesito vem se destacando o uso de jogos e atividades lúdicas, pois os ensina de maneira criativa e os motiva a descobrir mais. Banco Químico foi o jogo criado pelos autores, foi baseado no famoso jogo de tabuleiro Banco Imobiliário, o jogo criado pelos autores combina tabuleiro, cartas e dados e seus jogadores compram e constroem instalações ao longo do jogo, semelhante aquele que ele foi baseado, a diferença é que aqui todos os componentes foram modificados para que possuíssem uma temática relacionada a química, assim, é possível ver, por exemplo, ruas com nomes tipos viela chuvas ácidas, cartas sobre atuação técnica e profissional do espaço de trabalho dos químicos e etc.

Space Hunter, ou Gezegen Avcıları como também é conhecido no seu país de origem, foi criado por Kirikkaya et al (2010), é um jogo de tabuleiro projetado para avaliar o conhecimento dos alunos sobre os objetos celestes, o universo, pesquisas sobre o espaço, ferramentas utilizadas para pesquisar e observar o espaço e sobre as características dos planetas. O jogo foi projetado para 5 jogadores e, ao longo dele, os jogadores vão respondendo perguntas e para cada pergunta certa um deles ganha uma peça de quebra-cabeça, aquele que montar primeiro ganha.

Segundo a ideia desenvolvida por dos Santos et al (2016) é necessário que a escola e os professores se adaptem a esta nova realidade digital, pois hoje os alunos têm uma forma diferente de se relacionar com o conhecimento e possuem uma grande relação com a tecnologia, assim, com o intuito de incentivar a aprendizagem de artes no ensino médio, foi desenvolvido um jogo de tabuleiro, a partir do qual está sendo elaborada uma versão digital, chamado “Conquiste a obra”. O tabuleiro do jogo se baseia no mapa da cidade de Porto Alegre e destacam as regiões em que as obras e monumentos históricos estão localizados. O objetivo dos jogadores é conquistar as obras históricas que são representadas como peças no tabuleiro, ao longo das partidas, os jogadores irão coletando cartas com informações importantes sobre essas obras e aquele que obter determinadas informações conquista uma peça e ganha aquele jogador que adquirir mais obras.

\section{Proposta de Jogo}

O Jogo “Lendas do Brasil” será um jogo de tabuleiro digital que possuirá cartas com curiosidades do folclore, o conteúdo dessas cartas será relativo ao folclore nacional, possuindo curiosidades tanto conhecidads como desconhecidas, contando origem de mitos, festas e tradições, tudo isso afim de despertar o interesse do jogador a aprender mais sobre o assunto.

De maneira simples, no jogo os usuários deverão percorrer os espaços de um circuito, cumprindo determinações que alguns espaços espalhados pelo tabuleiro exigem e puxando cartas que podem auxiliar ou não as suas jornadas, tudo isso envolto de uma temática folclórica. Vence o jogo, o jogador que primeiro completar o circuito.

\section{Considerações Finais}
Este texto apresentou alguns conceitos fundamentais da criação de jogos, sejam eles sérios ou apenas lúdicos, sejam eles digitais ou físicos, tendo apresentado ainda uma análise de alguns jogos e o potencial que eles sustetam para o futuro projeto que será feito. Diante do que foi visto, fica claro a importancia dos conceito do jogos e do potencial que eles guardam e, com isso, é correto afirmar que mesmo jogos simples e casuais possuem uma grande capacidade não só em questões de entretenimento quanto também de incentivo a aprendizagem.

\begin{thebibliography}{1}

\bibitem{b1}Loureiro, Cezar Gigliotti, et al. "Design de jogos de RPG digitais: uma investigação sobre a experiência de jogo."

\bibitem{b2}Savi, Rafael, and Vania Ribas Ulbricht. "Jogos digitais educacionais: benefícios e desafios." RENOTE-Revista Novas Tecnologias na Educação 6.1 (2008).

\bibitem{b3}Benjamin, Roberto. "Conceito de folclore." Projeto Encontro com o Folclore (2011).

\bibitem{b4}Cavalcanti, Maria Laura. "Entendendo o folclore." Rio de Janeiro (2002).

\bibitem{b5}Delbem, Danielle Conte. "FOLCLORE, IDENTIDADE E CULTURA". 
    \url{https://github.com/JairtonF/Analisador-L-xico}. Acesso: 23 de Fevereiro de 2021.

\bibitem{b6}Brasil Escola. "Folclore brasileiro". 
    \url{https://brasilescola.uol.com.br/historiab/folclore-brasileiro.htm}. Acesso: 23 de Fevereiro de 2021.

\bibitem{b7}Fernadez, Florestan "O Folclore Em Questão". 2°ed., São Paulo. 2003.

\bibitem{b8}Lucchese, Fabiano, and Bruno Ribeiro. "Conceituação de jogos digitais." São Paulo (2009): 7.

\bibitem{b9}Crawford, C. (1982). The Art of Digital Game Design, Washington State University, Vancouver, 1982

\bibitem{b10}Pereira, Ricardo Francisco, Polônia Altoé Fusinato, and Marcos Cesar Danhoni Neves. "Desenvolvendo um jogo de tabuleiro para o ensino de física." Encontro Nacional de pesquisa em educação em Ciências, Florianópolis 8 (2009).

\bibitem{b11}Lucchese, Fabiano, and Bruno Ribeiro. "Conceituação de jogos digitais." São Paulo (2009): 7.

\bibitem{b12}LOPES, M. da G. Jogos na Educação: criar, fazer e jogar. 4º Edição revista, São Paulo: Cortez, 2001.

\bibitem{b13}Oliveira, Jorgiano Souza, Márlon Herbert Flora Barbosa Soares, and Wesley Fernandes Vaz. "Banco químico: um jogo de tabuleiro, cartas, dados, compras e vendas para o ensino do conceito de soluções." (2015).

\bibitem{b14}Kirikkaya, Esma Bulus, Sebnem Iseri, and Gurbet Vurkaya. "A board game about space and solar system for primary school students." Turkish Online Journal of Educational Technology-TOJET 9.2 (2010): 1-13.

\bibitem{b15}dos Santos, Míria Santanna, et al. "A combinação de jogos de tabuleiro com jogos digitais no processo de aprendizagem." Simpósio Brasileiro de Jogos e Entretenimento Digital (SBGames) (2016).

\bibitem{b16} Chagas, Maria Das Graças & Baère Pedrazzi Lomba de Araujo, Bruno & Faria, Waldecir & Loureiro, Cezar & Carmo, Maria & Donato, Nelson & Silva, Tathiana & Maidantchik, Victoria & Marques, Yann. (2015). Design de Jogos de RPG Digitais: Uma investigação sobre a experiência de jogo. 

\bibitem{b17} Sakai, Amanda. “Importância do Folclore” (2017).
 \url{https://medium.com/tfg-sophiakraenkel-2017/import%C3%A2ncia-do-folclore-f18c597e484f}. Acesso: 21 de Fevereiro de 2021.

\bibitem{b18}Falkembach, Gilse Antoninha Morgental, Marlise Geller, and Sidnei Renato Silveira. "Desenvolvimento de Jogos Educativos Digitais utilizando a Ferramenta de Autoria Multimídia: um estudo de caso com o ToolBook Instructor." RENOTE-Revista Novas Tecnologias na Educação 4.1 (2006).

\bibitem{b19}Juul, J., Half-Real: Video Games between Real Rules and Fictional Worlds, The MIT Press, 2005, ISBN: 0262101106

\bibitem{b20}La Carretta, Marcelo. Como Fazer Jogos de Tabuleiro: Manual Prático. 2018.

\end{thebibliography}



\end{document}
